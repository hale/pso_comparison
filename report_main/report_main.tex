\documentclass{csfourzero}

\author{Philip Hale}
\title{A comparison of developments in PSO from 1995 to 2013}

\begin{document}

\begin{titlepage}
  \maketitle
  \begin{abstract}
  This paper compares the industry-best standard of Particle Swarm Optimization
  (SPSO-2011) with unmodified canonical PSO that is close to it's original
  implementation first proposed in 1995. The success of each algorithm is
  determined by its ability to optimize NUMBER_OF_FUNCS functions. We begin by
  introducing PSO and explaining its historical limitations, and conclude by
  evaluating the extent to which these have limitations have been addressed by
  the current-best standard.
  \end{abstract}
  \tableofcontents
\end{titlepage}

\section{Introduction}

Particle Swarm Optimization is a comparatively new evolutionary computation
technique, having only been around since 1995.  Conceptually the algorithm
emulates a flock of birds circling a roost, resulting in  behaviour described by
the original paper as ``flying potential solutions through hyperspace,
accelerating toward `better' solutions.'' (Eberhart 1995).

Since the algorithm in its simple form is competitive with more complex and
mature optimization techniques, it has been the subject of much study since
then. These developments are primarily concerned with avoiding the potential for
the algorithm to prematurely converge on a non-global minima, and increasing the
speed at which convergence occurs.

Fifteen years later, a number of 

\subsection{Algorithm definition}

PSO ALGO DEFINITINO GOES HERE

\section{Key Developments}

Since shortly after its introduction some fundamental issues with PSO have been
apparent, namely premature convergence (SRC) and a comparatively slow convergence
speed (SRC).

\subsection{Canonical PSO}

Developments that led from original PSO to the canonical PSO analysed in this
paper.

\subsection{a thing in 2011}
Developments leading to thing

\subsection{another thing in 2011}
Developments leading to other thing

\subsection{another other thing in 2011}
Developments leading to another thing

\section{research question}

\subsection{Canonical PSO}

\subsection{SPSO 2011}

particles: 40 suggested value
initialization: adaptive random topology. Particular case of 'stochastic star'





\end{document}
